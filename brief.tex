%%% LaTeX Template
%%% This template is made for project reports
%%%	You may adjust it to your own needs/purposes
%%%
%%% Copyright: http://www.howtotex.com/
%%% Date: March 2011

%%% Preamble
\documentclass[paper=a4, fontsize=11pt]{scrartcl} % Article class of KOMA-script with 11pt font and a4 format
\usepackage{fontspec}
\usepackage[ngerman]{babel}				% English language/hyphenation
\usepackage{amsmath,amsfonts,amsthm}	% Math packages
\usepackage{graphicx}					% Enable pdflatex
\usepackage{url}
%%% Custom headers/footers (fancyhdr package)
\usepackage{fancyhdr}
\usepackage{csquotes}

\pagestyle{fancyplain}
\fancyhead{}							% No page header
\fancyfoot[L]{\small \url{nibelungen-siegen.de}}% You may remove/edit this line 
\fancyfoot[C]{}							% Empty
\fancyfoot[R]{\thepage}					% Pagenumbering
\renewcommand{\headrulewidth}{0pt}		% Remove header underlines
\renewcommand{\footrulewidth}{0pt}		% Remove footer underlines
\setlength{\headheight}{13.6pt}

%%% Equation and float numbering
\numberwithin{equation}{section}		% Equationnumbering: section.eq#
\numberwithin{figure}{section}			% Figurenumbering: section.fig#
\numberwithin{table}{section}			% Tablenumbering: section.tab#


%%% Maketitle metadata
\newcommand{\horrule}[1]{\rule{\linewidth}{#1}} 	% Horizontal rule

\title{
		%\vspace{-1in} 	
		\usefont{OT1}{bch}{b}{n}
		\normalfont \normalsize CDStV Nibelungen zu Siegen im Wingolfsbund \\ [25pt]
		\horrule{0.5pt} \\[0.4cm]
		\huge Stadtführung \\
		\horrule{2pt} \\[0.5cm]
}
\author{
		\normalfont 				\normalsize
        Koneo\\[-3pt]				\normalsize
        \today
}
\date{}


%%% Begin document
\begin{document}
\maketitle
\section{Siegen}
Siegen Kreisstadt des Kreises Siegen-Wittgenstein im Regierungsbezirk Arnsberg im Land Nordrhein-Westfalen. 31. Dezember 2014 wurde die Marke mit 100.325 Einwohnern erneut überschritten und gilt seitdem wieder als Großstadt. Sie ist die Geburtsstadt des berühmten Barockmalers Peter Paul Rubens, weshalb sich Siegen als Rubensstadt bezeichnet. Seit Juli 2012 nennt sich Siegen Universitätsstadt.
Die räumlichen Entfernungen zu den umliegenden Großstädten betragen in etwa 
\begin{itemize}\setlength\itemsep{0pt}
	\item 65 km (Hagen), 
	\item 95 km (Frankfurt), 
	\item 65 km (Koblenz) und 
	\item 75 km (Köln)
\end{itemize}
Siegener behaupten gerne, dass Siegen auf sieben Bergen gebaut ist, so wie auch Rom. Tatsächlich sind es mindestens acht Berge, die den Talkesseel umringen:
\begin{itemize}\setlength\itemsep{0pt}
	\item Giersberg (358 m), 
	\item Siegberg (307 m),
	\item Lindenberg (373 m),
	\item Häusling (364 m), 
	\item Rosterberg (326 m),
	\item Fischbacherberg (371 m), 
	\item Wellersberg (346 m) und 
	\item Heidenberg (315 m). 
\end{itemize}
 Im Mittel liegt Siegen auf 290 m über NN. Neben dem Berganteil ist auch der Waldanteil hoch 51\% der Fläche ist Wald. Mehr als ein Meter Regen fallen im Jahr, April ist am trockensten. Durchschnittlich ist es acht Grad warm, der Juli mit 17 Grad im Durchschnitt am Wärmsten.
 
 \subsection{Name}
 Der Name Siegen geht auf den vielleicht keltischen Flussnamen der Sieg zurück, ungewiss ist eine Verwandtschaft mit dem Namen des keltisch-germanischen Stammes der Sugambrer, der in vorchristlicher Zeit in Teilen des heutigen Nordrhein-Westfalen lebte. Die Freie Burschenschaft Sigambria entnimmt vermutlich daher ihren Namen, der auch für die Kuhreichen oder die Kuhgieren steht. Die Landschaft hier gibt aber wenig Platz für Kühe, die Buschafter müssen also wahre Ochsen sein.\par 
 
 \subsection{Eisenverarbeitung hat lange Tradition}
 Jedenfalls gab es hier von 500 v. Chr. bis 100 n. Chr  Eisenverhüttung für den Zeitraum von etwa 500 v. Chr. bis 100 n. Chr. nachgewiesen. Dann gab es hier kein Wald mehr und erst im Frühmittelalter (10./11. Jahrhundert) ging es mit der Eisenverhüttung wieder los.\par
 
 \subsection{Stadt Siegen und Haus Nassau}
 Eine Münze wurde gefunden und auf 1175 datiert, die von Siegen als \enquote{civitas} spricht. 1224 wurde Siegen erwähnt als eine aufs Neue erbaute Stadt. Diese übertug Erzbischof Engelbert I. von Köln dem Grafen von Nassau, Heinrich dem Reichen, zum halben Miteigentum übertrug. Es ist belegt, dass das Obere Schloss zu dieser Zeit schon existierte. 1303 erhielt die Stadt das Soester Stadtrecht. Bis 1381 hatte die Stadt zwei Eigentümer, dann ging sie gänzlich in die Hände der Nassauer über.\par
 Das Haus Nassau ist ein weit verzweigtes deutsches Adelsgeschlecht von europäischer Bedeutung. Ihm entspross der römisch-deutsche König Adolf von Nassau, auch der 1689 bis 1702 König von England, Schottland und Irland, das heute regierende niederländische Königshaus sowie das großherzogliche Haus von Luxemburg. Ihr jüngstes Mitglied ist ihre Königliche Hoheit Catharina-Amalia van Oranje. 
 
 \subsection{Ereignisse in Siegen ab 1536}
 Die Stadt Siegen bot im 16. Jahrhundert einen wehrhaften Anblick. Sie war von gewaltigen Mauern mit 16 Türmen und drei Stadttoren umgeben und besaß eine mächtige Burg. Die drei Tore der Siegener Stadtbefestigung waren das Kölner Tor nach Westen, das Löhrtor nach Süden und das Marburger Tor nach Osten. 
 \begin{itemize}\setlength\itemsep{0pt}
 	\item [1536] in den Gebäuden des ehemaligen Franziskanerklosters ein Pädagogium ein, aus dem das Gymnasium am Löhrtor der Stadt Siegen hervorgegangen ist. Sie war eine Hochburg der calvinistisch-reformiert geprägten Föderaltheologie.  
 	\item [1593] ein Großbrand vom Schmied Johann Busch in der Marburger Straße An der Stelle eines alten Franziskanerklosters errichtete er das Untere Schloss.Sein Sohn Johann der Jüngere trat 1612 wieder zur katholischen Kirche über und wollte dies mit Gewalt auch für die Bürger erzwingen. Johann Moritz von Nassau-Siegen, der holländische Befehlshaber in Brasilien, setzte ihn ab und es kam 1650–1651 unter seiner Regierung zu einer Teilung des Siegerlands nach Konfessionen. 
 	\item [1616] wurde eine ritterliche Kriegsschule im heute noch bestehenden alten Zeughaus an der Burgstraße eingerichtet.
 	\item [1695] ein Großbrand vom Bäcker Johann Daub in der Barstewende
 	\item [1673] versetzte ein Erdbeben die Bevölkerung in Angst und Schrecken.
 	\item [1699] zu Gewalttaten zwischen beiden Konfessionen. 
 	\item [1743] die katholische Herrscherlinie in Nassau-Siegen. Da mit Friedrich Wilhelm bereits 1734 auch die reformierte Linie erloschen war, übertrug Kaiser Karl VI. dem Prinzen von Oranien und Fürsten von Nassau-Diez die Regierung. Siegen war nunmehr Hauptort des Fürstentums Siegen innerhalb von Oranien-Nassau.
  	\item [1806] Der Bergbau im Siegerland, die Hauptquelle des Wohlstandes und die Landwirtschaft entwickelten sich positiv. Als Fürst Wilhelm von Oranien sich weigerte, dem von Napoleon initiierten Rheinbund beizutreten, wurde er von ihm abgesetzt. Das Siegerland wurde Teil des Siegdepartements innerhalb des Großherzogtums Berg. Nach der Niederlage Napoleons in der Schlacht von Leipzig nahm Wilhelm Friedrich als Prinz von Oranien im Dezember 1813 wieder Besitz seiner deutschen Erblande, die er aber 1815 an Preußen abtrat, wofür er im Gegenzug das Großherzogtum Luxemburg erhielt. Die Stadt wurde dem Kreis Siegen, zunächst im Regierungsbezirk Koblenz (Provinz Großherzogtum Niederrhein), ab 1817 im Regierungsbezirk Arnsberg (Provinz Westfalen) zugeordnet.
 	\item [1809] wird berichtet, dass es in Siegerländer Kirchen wieder Te Deum als Dank für Napoleons Erfolge gab.
	\item [1813] fand im Siegerland ein vier Tage andauernder Durchzug von 16.000 Soldaten Blücher smit 8.000 Pferden statt. 
	\item [1814] konnte Siegen erstmals eine Straßenbeleuchtung in Form von 16 Petroleumlampen aufweisen
	\item [1841] trat in Siegen der erste \enquote{Königliche Land-Fußboden-Postler seinen Dienst an} – Der Anfang des Internets
 	\item [1869] ein Großbrand vom \enquote{Klubb}
 	\item [1875] war der Tag des erstmaligen Einsatzes einer von einem Pferd gezogenen Straßenfegemaschine auf der Sandstraße und der Koblenzer Straße in Siegen.
 	\item [1881] richtete ein Unwetter nach wochenlang anhaltender Hitze schwere Schäden in der Stadt an.
 	\item Durch den Anschluss an Preußen wurden die historischen Bindungen nach Süden aufgelöst. Das Siegerland wurde nach Westfalen hin ausgerichtet, von dem es bis dahin durch eine jahrhundertealte politische, kulturelle, sprachliche und konfessionelle Grenze getrennt gewesen war.
 	\item [1918] Zur Novemberrevolution konstituierte sich auch in Siegen ein Arbeiter- und Soldatenrat. Er setzte sich die Aufgabe, für \enquote{Ruhe, Ordnung und Sicherheit} zu sorgen.
 	\item [1929] bestand der Rat der Stadt Siegen aus neun Mitgliedern der DNVP, acht der Zentrums und jeweils vier von DVP, SPD und NSDAP. Die restliche vier Sitze belegten \enquote{Sonstige}.
 	\item [1932] war die NSDAP (45,6 Prozent; Preußen: 36,3 Prozent) die mit weitem Abstand stärkste Partei. Die zweite Position hatte zu diesem Zeitpunkt das Zentrum (18,4 Prozent).
 	\item [1933] Auf die Übergabe der Regierungsgewalt an die NSDAP und ihre deutschnationalen Bündnispartner \enquote{Kabinett Hitler}) am 30. Januar 1933 folgten in Siegen zunächst die Schließung des Parteibüros der KPD, Hausdurchsuchungen und Beschlagnahmungen, dann Verhaftungen, nach den bereits unter irregulären Bedingungen stattfindenden Wahlen am 5. März systematische Verhaftungsaktionen, ferner Verschleppungen von Mitgliedern und Anhängern von KPD, SPD, Zentrum und Freien Gewerkschaften in den Keller des Braunen Hauses der NSDAP, wo sie misshandelt und gefoltert wurden.
 	\item [1933] begann im Siegerland die Einrichtung von Arbeitslagern.
 	\item [1938] wurde die Synagoge der Jüdischen Gemeinde im Obergraben von einer Gruppe Nationalsozialisten, die meisten Angehörige der SS, unter den Augen einer großen Zahl Schaulustiger verwüstet und in Brand gesetzt. 
 	\item [1940] begann in Siegen, Weidenau und Geisweid der Bau von 16 Luftschutzbunkern. Entsprechend dem Vormarsch der Wehrmacht wurden seit 1939 ausländische Zwangsarbeiter in die Stadt gebracht. Kurz vor dem Höhepunkt des Arbeitseinsatzes in der ersten Jahreshälfte 1944, waren 2.310 Menschen aus neun Staaten in 22 Siegener Lagern und in einigen wenigen Privatquartieren gemeldet. Knapp zwei Drittel der Zwangsarbeiter stammten aus der Sowjetunion. 141 davon waren Kinder unterschiedlicher Altersgruppen.
 	\item [1942] kam es zur ersten Deportation jüdischer Siegener. Wenige überlebte. Das Eigentum der Deportierten wurde enteignet und ging in den Besitz der Mehrheitsbevölkerung und des Staates über.
 	\item [1944] ging ein Transport von mit Nichtjuden Verheirateten sowie von \enquote{Mischlingen 1. Grades} zur Zwangsarbeit in verschiedene Lager.
 	\item [1944] Am 16. Dezember 1944 wurde das Stadtzentrum Ziel eines schweren britischen Luftangriffs bei dem 80 Prozent des Stadtgebiets zerstört wurden. Damit war der Luftkrieg über dem Siegerland erst spät in seine dramatische Phase eingetreten. Die Folgen verlief vergleichsweise glimpflich, da bei frühen Kriegsvorbereitungen viele Schutzräume gebaut wurden und der Bevölkerung bei den Luftangriffen Schutz boten. Die Bevölkerung zum Widerstand gegen die Führung zu bringen, bewirkte der Luftkrieg auch im Siegerland nicht.
 	\item [1945] Am 6. April 1945 gelang es den US-Truppen die Kasernen am Westrand der Stadt einzunehmen, drei Tage später konnten sie ihren Gefechtsstand in Siegen schließen. Damit war der Krieg für die Stadt beendet. Insgesamt wurden von den vor Kriegsbeginn in der Stadt vorhandenen 4.338 Gebäuden mit 10.452 Wohnungen über 90 Prozent, nämlich 4.096 Gebäude mit 10.169 Wohnungen, beschädigt oder zerstört. Alle Brücken über die Sieg waren von der Wehrmacht gesprengt worden. Allein in Eiserfeld wurden am 30. und 31. März 1945 fünf Eisenbahn- und fünf Straßenbrücken zerstört.
 	\item [2004] entstand in einem Siegener Wohngebiet am Rosterberg ein Tagebruch. Häuser waren einsturzgefährdet. Dieser Bergschaden wurde als Siegener Loch bekannt.
 \end{itemize}
 
 \subsubsection{Demarkationslinien durch Siegen}
 \begin{itemize}\setlength\itemsep{0pt}
 	\item Kontraste ziehen sich durch Siegen: 
 	\item Aldi Nord vs Aldi Süd
 	\item 30 Jährige Krieg, Oberes Schloß versus Unteres Schloß (katholisch)
 	\item Banditos versus Hells Angels
 \end{itemize}

\section{Sieghütte} 
Sieghütte ist ein Siegener Begriff und kennzeichnet ein Gebiet, welches über mehrere städtisch benannte Bezirke geht. 

\section{Hagener Straße}
Rotlichviertel. 

\section{Kaisergarten}

\section{Oberes Schloß}
Die Höhenburg des Oberen Schlosses auf dem Siegberg wurde 1259 erstmals urkundlich erwähnt und war im Mittelalter die Stammburg des Hauses Nassau. Seit 1905 ist dort das Siegerlandmuseum untergebracht.

\section{Nikolaikirche}
Das Krönchen ist das Wahrzeichen der westfälischen Stadt Siegen. Die mehrere Meter hohe Plastik ist eine Kunstschmiedearbeit aus vergoldetem Eisen. Sie stellt eine überdimensionale Krone mit Windpfeil und Windrose dar. Seit 1658, dem Jahr seiner Stiftung, stand das Original des Krönchens auf der Spitze des Turmhelms der evangelischen Nikolaikirche in Siegen. Im Jahr 1993 wurde es aus konservatorischen Gründen durch ein Replikat aus ebenfalls vergoldetem Edelstahl ersetzt.\par

Das Krönchen ist ein Geschenk von Fürst Johann Moritz zu Nassau-Siegen an die Stadt Siegen und ihre Bürger. Der Anlass für das Geschenk war die Erhebung von Johann Moritz in den Fürstenstand im Jahre 1652. Der Fürst ließ die Plastik auf eigene Kosten im Hammer vor der Hardt in Weidenau von den drei Schmieden Gerlach Burchmann, Jakob Schleifenbaum und Johannes Pickardt anfertigen.

\section{Rathaus}

\section{Altstadt}

\section{Poststraße}
Fußgängerzone Alte Poststraße (Oberstadt) mit der am 24. November 1983 eingeweihten, von Wolfgang Kreutter geschaffenen Brunnenanlage.

\section{Unteres Schloß}
Am Ende des 17. Jahrhunderts entstand das Untere Schloss in seiner heutigen, einem offenen Rechteck gleichenden Bauform und diente der evangelischen Linie des Hauses Nassau-Siegen als Residenz. Zum Schloss gehört auch der Dicke Turm mit Glockenspiel. Im Jahr 1959 richtete die Stadt Siegen im Innenhof des Schlosses eine Gedenkstätte für Opfer von Krieg und Gewaltherrschaft ein. Im Schloss befindet sich auch die Gruft des evangelischen Teils des Nassauischen Fürstenhauses.  Das Untere Schloss diente als Landesbehördenhaus, in dem der Bau- und Liegenschaftsbetrieb NRW, die Bezirksregierung Arnsberg, Außenstelle Siegen, das Amt für Arbeitsschutz (bis 2012) das Arbeitsgericht Siegen (bis 2014), und – von 1936 bis zum 17. Januar 2011 – die Justizvollzugsanstalt Attendorn, Zweiganstalt Siegen ansässig waren.[58] Nach einer Zwischennutzung durch Forschungseinrichtungen der Universität Siegen in den Jahren 2012–2013[59] finden seit 2014 Sanierungsarbeiten für eine Nutzung durch die Universität Siegen statt. Auf dem Schlossplatz des Unteren Schlosses finden regelmäßig kulturelle Open-Air-Veranstaltungen statt. Zur Fußball-Weltmeisterschaft 2006 zog die Übertragung der Spiele auf einer Großleinwand bis zu 10.000 Zuschauer an.

\section{Siegbrücke}
Henner und Frieder sind zwei in der westfälischen Stadt Siegen stehende, vom Bildhauer Johann Friedrich Reusch geschaffene Statuen. Die Skulpturen aus Bronzeguss stellen einen Bergmann (\enquote{Henner}) und einen Hüttenmann (\enquote{Frieder}) dar.\par
In der Stadt Siegen wurden die beiden Figuren am 28. September 1904 auf den extra dafür verstärkten Mittelpfeilern der 1882 errichteten Siegbrücke aufgestellt. Dort standen sie auf Sockeln aus rotem Sandstein, die der Steinmetzmeister Kögler errichtet hatte. Die beiden Sockel waren auf der Vorderseite jeweils mit einem Sinnspruch als Inschrift versehen; auf den Seiten der Sockel waren die Worte „Geschenk der Aussteller des Siegerlandes, Düsseldorf 1902“ sowie „Errichtet von der Stadt Siegen 1904“ eingemeißelt. Der Sinnspruch am Statuensockel „Frieders“ lautete „Arbeit ist des Bürgers Zierde, Segen ist der Mühe Preis“. Zu Füßen „Henners“ war zu lesen „Wer Bergwerk will bauen, muß Gott vertrauen“. In den 1930er-Jahren mussten beide Skulpturen ihren Platz vorübergehend wegen des Neubaus der Siegbrücke räumen, die durch eine Stahlbetonbrücke an gleicher Stelle ersetzt wurde. Noch vor der im März 1934 erfolgten Brückenfreigabe kamen sie im Januar des gleichen Jahres an ihren alten Standort zurück, dieses Mal auf einfachen Fundamenten aus Stein.\par
Mit der Sprengung der Brücke durch die Wehrmacht am 31. März 1945 wurden die beiden Skulpturen gegen Ende des Zweiten Weltkrieges in den Fluss Sieg befördert. Sie wurden erst nach einigen Wochen geborgen, um anschließend im Eintracht-Bunker zwischengelagert zu werden. Nach erfolgter Restaurierung der durch die Brückensprengung erheblich beschädigten Figuren führte ihr Weg ein drittes Mal zur Brücke zurück. Dort wurden sie auf die Brückenköpfe zur Heeserstraße beziehungsweise zur Siegstraße gesetzt. Wegen erneuter Bautätigkeit im Umfeld der Brücke, etwa bei der Errichtung des dortigen City-Kaufhauses und dem Bau der Fluss-Überkragung Siegplatte wurde ihr Standort noch mehrfach geringfügig in der näheren Umgebung verlegt. Bis zum Sommer 2012 standen sie nebeneinander am Rande der Siegplatte an der dortigen Bahnhofsbrücke.

%%% End document
\end{document}